%%%%%%%%%%%%%%%%%%
%%% To be filled %
%%%%%%%%%%%%%%%%%%

\newcommand{\yourtitle}{Dissertation Title, that can span    %
                        over multiple lines if needed}       % The title of your thesis
\newcommand{\yourname}{First Last}                           % Your name
\newcommand{\youradvisor}{Dr.\ Advisor}                      % The name of your advisor(s).
% The ".\ " makes sure the spacing after the dot is correct, cf. https://tex.stackexchange.com/a/2230/.
\newcommand{\yourkeywords}{Key1, Key2, A longer keyword}     % Separate your keywords with ",".
\togglefalse{ms}                                             % Comment this toggle if you are a MS student.
\newcommand{\yourdate}{2024-12-23}                           % Enter the date in the YYYY-MM-DD format. Only the month and year will be displayed on the cover 

%%%%%%%%%%%%%%%%%%%%%%%%%
%%%% Optional arguments %
%%%%%%%%%%%%%%%%%%%%%%%%%

% Adding a precise licence will make your work
% easier to share, cite and adapt.
% You can use
% https://guides.augusta.edu/c.php?g=796891&p=6003854
% or 
% https://www.aje.com/arc/creative-commons-intro/
% to help you decide on a licence.
\newcommand{\yourlicence}{\href{https://creativecommons.org/licenses/by/4.0/}{CC Attribution 4.0 International}}    % More precise licence.
% To pick a licence, you can use e.g. https://beza1e1.tuxen.de/licences/ or https://choosealicense.com/ to help you decide.

\newcommand{\yoursubtitle}{Subtitle (Optional)}                                                                     % The subtitle.

% You can add a "mention", typically to 
% indicate that your manuscript is a draft
\newcommand{\yourmention}{Draft} 

